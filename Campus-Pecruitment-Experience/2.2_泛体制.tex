\section{泛体制(国企、选调、事业单位等)}

% \subsection{关于面试(王世聪)}

% % \section{面试(AI面/群面/无领导小组讨论/技术单面/结构化单面/英语面/HR面/主管面)(王世聪)}




% 我们在校招期间会面临的面试大体可以分为两大类:非技术类和技术类。在泛体制类单位的面试中,不会有大厂那样很刨根问底的技术面试,通常问问简历项目和一些基础问题就结束,甚至有的都干脆不问,更重要的是一些非技术类面试。我定义的非技术类的面试包括泛体制内的结构化面试、无领导小组讨论/辩论赛、普通群面(多对一)、单面(如企业中的HR面、主管面等)。技术类的面试则涵盖算法题、项目考察、技术基础知识(八股文)、业务场景题等。


% % 面试归根结底是人和人的互动,尤其是非技术面,表达诚意,展示热情,可能比回答的内容更重要,相较于把它当成一次\textbf{考试},更合适的心态是把它当成一次\textbf{相亲}






% \subsubsection{结构化面试}


%  结构化面试通常出现在公务员选调及一些国企事业单位的面试中。这类面试通常给出一个开放性的问题,提供一定的思考时间,让应试者展开回答。题型通常包括组织规划类、观点明晰类、人际交往类等。



% 针对结构化面试的应试策略可以分为两种:


% 一种我称之为\textbf{素材梳理式},适合绝大多数没有接触过行政话语的同学,因为按照往年各机构整理出的参考答案,可以发现,其实每一类题型都可以整理出对应的模板,里面嵌入的内容也有共性,比如组织规划类,经常出现事前要充分调研,事后要分析反馈,事中要维持秩序,做好应急预案等,那这些一个个通用的举措,就是我们要积累(背下来)的素材,再根据具体的题目情景进行细化的表述,就形成了一个各个环节紧密耦合,非常细致且全面的回答。这种方式需要通过大量的针对例题的实战练习,逐渐打磨出一套适合自己的答题模板,并积累内容素材和套话。练的特别熟悉之后,在正式面试时,通过快速反应、流畅表达和全面的内容输出给面试官留下深刻印象。这种方式的重点是练习时要掐时间脱稿,尽可能模拟面试现场情景,以克服紧张情绪,改掉不良口癖和动作,找到适合自己的表达方式。这样准备面试的同学,可以关注复旦大学的基层就业协会,它们每年都会在秋招期组建各个省的复旦官方选调群,复试期间里面会有复旦的同学们一起约线下对练,可以找一些同省考试的同学一起面对面练习,这样能听到对方给你的反馈意见,相互纠错。


% % 另外一种我称之为\textbf{观点爆破式},适合平时就非常关注时政议题,有大量的观点累积,同时具有一定的思辨和表达能力的同学。这种方式需要我们突出重点,输出一个漂亮的观点,展示我们思考的深度。试想一下,在面你之前面试官可能都听了一天的模板,那些相同的套话和素材可能都听麻了,而这个时候你出场,从一个新颖的角度,细致阐述了一个有思考的论点,就会让面试官眼前一亮,脱颖而出。还用组织规划类的题型举例,这种策略的叙述格式类似于:虽然组织这个活动需要事前调研,准备,事后分析等等诸多环节,但是在这个活动里最关键/核心问题/最需要把握的环节其实是(比如说要做好紧急预案),因为这个活动具有某些鲜明的特点(比如大夏天在户外容易中暑啊,或者人员密集聚集容易踩踏,或者是线上直播容易有舆情之类的),曾经就有类似的活动因为没有注重这类隐患而出现了严重的问题(举例子),所以我们该吸取前车之鉴,做好以下工作(阐述举措)。所以梳理式是熟练有结构的罗列素材(类似BFS),爆破式是针对一个环节/观点进行深入阐述(类似DFS),这种准备的话需要同学们关注对应省份近一年来的重要议题,想一些自己的观点,并且记录下来,有新想法,或者在网上遇到了相关的案例,就把之前的记录更新,这样就能逐渐积累出自己的观点和案例库,增加现场面试时输出观点的机会。



% 另外一种我称之为\textbf{观点爆破式},适合平时关注时政议题、拥有大量观点积累并具有一定思辨和表达能力的同学。这种方式需要我们突出重点,输出一个漂亮的观点,展示我们思考的深度。
% 试想一下,在面你之前面试官可能都听了一天的模板,那些相同的套话和素材可能都听麻了,而这个时候你出场,从一个新颖的角度,细致阐述了一个有思考的论点,就会让面试官眼前一亮。
% 例如,回答组织规划类题型时,可以从活动的关键环节入手,指出最需要关注的地方(比如说虽然组织这个活动需要xxx,xxx等诸多环节,但是在这个活动里最关键/最核心问题/最需要把握的环节其实是要做好紧急预案),然后分析活动特点(如户外活动容易中暑、人员密集容易踩踏、线上直播易引发舆情等),并结合实例进行说明(比如曾经就有类似的活动因为没有注重这类隐患而出现了严重的问题),接下来再介绍举措并解释理由,最后几句话带过其他不重要的环节。
% 这类策略在准备上要求同学们关注报考省份近一年来的相关议题,形成自己的观点,并且记录下来。之后有新想法,或者在网上遇到了相关的案例,就不断更新之前的观点和案例库,以备面试时灵活应对。


% 总的来说,梳理式类似广度优先搜索(BFS),强调熟练的、完整的且结构化的素材罗列,可以短时间内速成;爆破式则类似深度优先搜索(DFS),着重深入阐述某一环节或观点,需要一些基础能力和日常积累。根据自身特点选择合适的应试策略,才能在面试中脱颖而出。

% \subsubsection{无领导小组讨论/辩论赛}






% 无领导小组讨论/辩论赛的面试常出现在国企和部分省份的选调,一些大厂的非技术岗也有这种形式的面试,有些还混合着来,讨论中间掺杂辩论环节,最后再进行总结,这种我们其实在校园里有过体验,求职面临这样的面试有几点经验:


% 1.切记克制情绪。这种面试不是为了吵赢其他的面试者,而是要把好的一面展示给旁听者,在和其他面试者们一起讨论甚至辩论的过程中肯定会有不同意见的碰撞,保持耐心倾听和礼貌表达,注意千万不要带着情绪,表现得过于强势,你一上头,旁观者就很下头,结果就是被挂。

% 2.注意遵守规则。很多学长姐反馈的经验中都提到,如果自己的发言环节有时间限制,一定不要超时,尤其是在一些选调面试中。如果时间到了,就算没说完也要停下来,避免超时。

% 3.争取有效输出。
% 有效输出是指能够给面试官留下深刻好印象的表现。在无领导小组讨论中,虽然会有领导者(leader)、计时员(timekeeper)等角色,但校招面试中,面试官通常不会特别在意这些角色标签。一场面试下来,讨论的质量可能并不高。在这种情况下,有两个动作会特别出彩:

% (1)总结:这是最有用的动作,总结也包括两部分,一个是面试的最后会要求选出一个人做总结汇报,这是一个备受关注的环节。如果能够争取到做总结的机会,并做好总结,会非常加分。即使无法担任最后的总结人,也要主动推举一个合适的人选,并给出充分的理由,避免在这个环节变成小透明。

% 另一个重要的部分是进行阶段性总结。当讨论变得混乱、大家提出许多不同角度的阐述时时,能够及时对内容进行概括和归纳,并向其他人确认,是非常有价值的。这种阶段性总结不仅可以使讨论更加清晰明了,还能引导大家进入下一个讨论环节。当面试官对讨论内容感到困惑时,他们往往会期待有人能够清晰地总结当前的讨论进展。这样的总结会让你很自然的成为推动讨论进程的人,并有效地掌控讨论的节奏。





% (2)观点靠近业务:提出与应聘单位和岗位的业务内容相关的观点,能给面试官留下深刻印象。在面试前,多准备一些关于应聘单位和岗位的信息。在表达观点时,尽量结合这些要素,让人感受到你对这个单位或者这个岗位的充分了解和重视,会有意想不到的收获哦。


% \subsubsection{单面/主管面}
% 主管面一般都是单面,单面是最常见且基础的面试形式,可能是一对一的面试,也有可能是多对一的形式。你可能会面对多个业务骨干和HR,也可能只是与一个部门主管进行交流,但不论如何,面试者始终只有你一个人。


% 无论是大厂还是央国企,主管面都是整个面试过程中最关键的一环。面试官通常是主管或领导,他们通过这轮面试来了解新人,不仅决定候选人是否能够获得offer,还可能影响最终的岗位分配。

% 由于面试官的身份,他们对候选人的评估具有一定的影响力。根据学长姐们的经验,如果你在面试过程中与面试官交流愉快,即使最终因为人岗不匹配等原因,而未能获得该岗位的offer,也可能带来意想不到的收获。例如,面试官可能会将你的简历推荐给其他部门或不同的岗位、城市base。甚至能提供更好的机会让你去尝试,比如一些原本BG无法接触到的或无法通过简历筛选的岗位机会。因此,值得大家好好去把握。笔者根据自己的面试经验,以及采访过中学长姐们给到的一些建议,总结成以下几点经验,希望能给学弟妹们提供一些有效的参考:

% \textbf{1.与其说是考试,其实更像相亲}



% 在这里将主管面比作相亲,是因为在这类面试中,面试官通常不会过多考察具体的知识和技能,而是更倾向于通过聊天的形式了解候选人。例如,他们会问一些类似“你做过的最有成就的一件事是什么”,或者“在面对压力时你通常怎么处理”这样的问题。通过这些问题,希望在对话中了解面试者是个什么样的人,以及了解一些你对于工作岗位的需求和看法,来判断一下你和岗位的匹配度。

% 在采访过程中,许多受访者提到,他们在面试时感觉自己回答得并不完美,但最终却收到了offer。入职了之后才发现,原来是当时的面试官觉得自己很特别。很多时候,这类面试的结果并不像考试那样单纯取决于回答的正确性,而更多取决于双方微妙的\textbf{情绪互动}。也不像考试一样是单向考察,而应该是\textbf{双向选择}。

% 一个理想的面试结果,是双方都聊得很愉快,并且互相留下了深刻的好印象。那么,如何才能实现这样的结果呢?实际上,这很大程度上取决于缘分。同样的表现,在不同的面试官眼中可能会产生截然不同的印象。因此,在面对这种不确定性时,我们需要做好自己能掌控的部分。笔者的建议是:真诚地展现自己独特的闪光点。


% \textbf{2.宁可清澈的愚蠢,别做拙劣的骗子}

% 会有一些同学对自己的故事没有信心,灌水一些高大上的虚拟经历去应付问题,或者是为了迎合自己认为的面试官喜好,做了很多违心的回答。建议不要这样!

% 一是我们作为学生,被学校规训了这么多年,其实很难扮演一个高明的骗子,尤其在面试过程中,我们处在一个不对等的压力下,乱讲时那个飘忽的眼神,颤抖的双手,都很容易暴露。人家老油条一看就知道你在乱讲,一旦人家怀疑你在撒谎,那基本就无了。

% 二是就算糊弄成功,人家信了给了你offer,


% \textbf{3.你是孤天里的鹤,而非写满字的纸}

% 两个例子
% 宁可清澈的愚蠢,也别做拙劣的骗子

% 真诚,表达诉求,尤其在实习

%真诚,不代表直白,更不是口无遮拦,真诚的表达,需要大量的技巧


%嵌入故事,悬疑,细节,角色,情绪

%跳出优绩主义,反诠释自己的经历。

%如非科班

%如最大的优势,最满意的一件事。




% 你不是写满字的纸,而是孤天里的鹤。


%






\subsection{三桶油}
\subsubsection{简历投递}

三桶油里中海油一般是最早出招聘公告的,秋招会在每年的8月底,中石油和中石化也会在9月出招聘公告。具体可以关注公众号“石油招聘网”,或每个公司自己的公众号。这里提一下春招,我们那一年春招的时候还是开了一些岗,据我观察有一些老牌单位是没有在春招放出岗位,所以有想去的单位还是要在秋招试一下。春招也有好处,就是竞争会小很多,据说秋招的报录比是几百比一,春招是几十比一。有认识的同学报中海油,差不多类型的岗位,秋招没有面试机会,春招就进面试了,所以如果有好的岗位可以在春招捡漏。

这里再提一个捡漏技巧,在9月统一报名放完岗位后,三桶油还会在官网补发一些通知放出新的岗位,我工作的岗位就是在9月底新增的,因为很多人不知道所以竞争压力小了很多,这就要求大家隔几天去官网上看看,尽量消除信息差。

简历投递时一定要去官方的招聘网站,三桶油都不会通过第三方招聘网站校招。
在招聘官网上,首先按专业搜索适合自己的岗位,计算机类的岗有人工智能算法、软件研究、网络安全、软件开发等,还有很多岗位在名称上没有明确的计算机特点,但是在专业要求里有计算机类,也是可以报的。这里需要注意的一点是,网站会实时显示每个岗的报录比,一般来说人工智能岗位的竞争是最激烈的,如果为了保稳可以在临近截止日期根据报录比考虑是否换岗。其实岗位之间的差异没有那么大,只是在笔面试的时候会有所不同(这点在后面会讲到),进入单位后会再次定岗,这届就有人工智能进来的校招生定岗到运维组、产品组、测试组。项目分下来也都是大家一起合作干一样的工作,所以不要在意岗位,选个报录比低的能稳稳进来最重要。

根据专业筛选完岗位后,会发现很多单位都招计算机,但按规定只能报两个志愿,这就到了单位的选择上。三桶油除了集团总部外,都有自己的强势单位、和不那么适合大家的单位,我在知乎上搜过中石油的,进来后只能说知乎说的很准,大家也可以在知乎上搜索看。好在这家单位在知乎上被提到的很多,大家投之前可以搜一下参考一下。

如果实在不知道选什么单位,这里也有大方向可以参考,就是未来的政策方向、赚钱的业务方向、起家的拿手方向。这里展开举例,国家政策要求国企数字化改革,那么数智化的研究单位就可以报。国家政策要求大力发展清洁能源,那么新能源的研究单位就可以报。起家的拿手方向,比如中石油在上游的油气勘探业务占优势,它的勘探研究院就属于基石。而中石化在下游的油气炼化业务占优势,它的炼化研究院就属于基石。

选好了单位和岗位,就可以投递了,这里需要注意的有两点。一是一定要再次确认自己的专业是否在岗位需求的专业列表里,如果专业不对很可能在简历关就被机筛卡掉,浪费一个志愿。二是要提前准备好所有需要的文件资料,这里以中石化为例,它需要六级英语证书原件的扫描版(有红色章),如果上官网成绩单截图,对面hr是不认的,他只认那个红章。如果报名资料或专业不符合要求,是会有短信提醒的,时刻关注短信,按要求修改就好。

\subsubsection{笔试}

中石油的笔试有两场,第一场是类似行测的性质,第二场是专业笔试。中石化真正的笔试只有一场

中石油的第一场笔试,QS排名靠前的学校会免笔试,所以不用准备。第二场笔试,要求双机位,我考试后看相册,发现后置机位会随机抓拍,所以还是很严格的。这里就涉及到了岗位选择上,之前提过入职后工作内容差异不大,选择报录比低一点的岗位会更稳,但前提是那个岗位的知识你要学过。我当时选择的是网络安全研究岗,整个笔试都是网络安全。前面的填选主要考察该专业的基础知识,后面的大题需要设计技术方案,需要一定的知识储备和项目经验。但是,我认识的同学都没有在这一轮笔试被淘汰过,所以放平常心尽力答题就好。

中石化真正的笔试只有一场,就是被QS靠前免去的行测。还会有一场不限答题时间的介于心理测试和综合测评之间的测试,当时一顿乱填过了,通过并不难。

\subsubsection{面试}

中石油的面试有两场,一场基础面,一场ppt面。中石化只有一场ppt面。

中石油的基础面十分钟,是你所报岗位的主任来面。会问最基础的八股,项目经历。然后最重要的,会问你这个方向的技术前沿动态,有很多人会卡在这里。我选的偏冷门的网络安全研究岗位,竞争没那么激烈,大概是8个人里,会有1个人进入下一轮面试。

中石油的ppt面则比较极限,一般是今天中午告知题目,明天早上8点前就要求把ppt做好发到hr邮箱,然后讲ppt,讲解限时15分钟。题目说实话很难,需要搜查大量资料,我当时是直接通宵了。这轮主要是把ppt做的美观,然后讲的时候自信流畅别磕巴,就基本成功了。这一轮是5个人里,有1个拿到offer。

这里提一下做ppt的要点,千万不要整页都糊满字,这种ppt做得像word一样的,基本就废了。每页只放一两行字说明该页的主要内容是啥,剩下的部分放图和表。在配色上,以暖色系为主,如果需要对比说明再涉及冷暖色系。如果能找到数据,最好量化,比如你想说明重要性,百分之多少的企业已经做了,肯定是比大部分企业已经做了更有说服力。最重要的是,最后一定要把技术落实在石油化工领域,比如你讲大模型,最后一定要说这个技术如何应用落地在中石油,能为中石油带来什么,而不是在那里纯讲技术,领导关心的永远不是技术,而是技术落地后能带来什么效益,但这就需要你真的了解这个单位的实际情况,这里提一个技巧,可以去搜该单位的公众号。我面试前不久,刚好中石油参与了一个网络安全大会,会上有大量的网络安全技术落地场景文章都发在公众号里,我觉得我面试时这绝对是加分项,因为面试官问了我是从什么渠道了解到这么多相关材料的,并表示了赞扬。

中石化的面试只有一场,是ppt面,也是限时15分钟。但是它没有限定题目,就要求把你上学阶段研究的成果做一个介绍汇报,更类似毕业论文答辩简短版,所以难度不大。但是它时间和国网面试撞车了,所以我并没有参加。这一轮面试是10个人里,有1个拿offer,但面试开始前大概有6个人在群里表示放弃面试,所以只要坚持到最后,就有很大的概率拿到offer。

\subsubsection{签约}
中石油每轮面试和出结果都间隔一周,如果一周还没收到消息基本就是凉了。二面一周后会有hr打电话谈薪谈福利,如果放弃就结束,如果接收就会签两方(两方违约也有违约金)。然后大约一月中下旬会在官网公示名单,没什么问题就可以签三方了。这里补充一点的是,中石油会等延毕的学生,在国企里属于极少数,这一届单位里有一个9月才能入职的,还有一个明年1月才能入职的,都给保留了岗位,有毕业压力的小伙伴可以考虑。


\subsection{公务员考试}
\subsubsection{报考}

\subsubsection{笔试}
\textbf{行测部分:}常识判断:注重日常积累,广泛涉猎政治、经济、法律、文化、科技等方面的知识。建议每天阅读官方媒体发布的新闻报道、重要文件和政策解读,增加知识面。这个模块短时间提升有些难度,需要日积月累。
言语理解与表达:通过大量练习题提高阅读理解能力和语言表达能力。特别注意积累词汇和成语,掌握常见语法结构和修辞手法。
数量关系:掌握基本的数学运算公式和解题方法,如等差数列、等比数列、工程问题、行程问题等。通过大量练习真题,总结解题技巧,提高解题速度。
判断推理:熟悉各种推理规则和逻辑关系,如图形推理、定义判断、类比推理、逻辑判断等。判断推理可以通过大量训练进行提高,可以多训练学习相应模块。
资料分析:学习统计术语和基本的计算方法,提高对数据的分析和处理能力。建议每天进行限时训练,提高计算速度和准确性。资料分析是可以通过训练有效提分的模块。建议多训练学习,学一些速算技巧,做到可以迅速解答。

\textbf{申论复习:}
申论最重要的是多进行训练,多进行复盘。练习一下书写,字迹工整可以给人更好的阅卷体验,正楷字体也需要多提升书写速度,申论的时间把握也非常重要。
阅读理解:多读申论材料,提高阅读速度和理解能力。学会抓住材料的关键信息和主旨,为后续的答题做好准备。
写作训练:多写申论文章,提高文字表达能力和逻辑思维能力。注意文章的结构、层次和语言表达,要符合申论的写作要求。建议每周至少完成一篇申论写作练习。
关注时事热点:关注社会热点问题、国家政策和法律法规的变化,了解相关的背景知识和政策导向。将这些内容融入到申论写作中,使文章更具时代性和针对性。

\textbf{答题技巧}
合理安排时间:根据考试时间和题型难度,合理分配答题时间。对于难度较大或分值较低的题目,可以适当缩短答题时间,优先保证能够拿到的分数。
学会取舍:如果遇到不会的题目或难以在短时间内解决的问题,不要过于纠结,要学会取舍,先跳过该题目,继续做后面的题目,以免浪费时间。多做几次模拟套题,提前把握时间节点,在考试中能更好的感觉自己的写题速度。

% \subsubsection{申论}

\subsubsection{面试}


\subsection{国家电网}

国家电网的架构为总部-大区分部,总部-省公司,总部-直属机构。其中公司总部近年来只招聘电气工程专业的毕业生,而其他的分部,省公司,直属机构都会招计算机类专业毕业的同学。国家电网每年招聘每年一共有三次,提前批,一批,二批,每年的招聘、录取通知都会在官方人力资源招聘网站上公告
~\url{https://zhaopin.sgcc.com.cn/sgcchr/static/home.html}。提前批一般会在每年的9-10月份出公告,一批会在11月左右,二批会在第二年的3月左右。提前批主要面向原985,211,原电力部属院校的电气类专业研究生,以及部分原985院校的\textbf{计算机类,通信类专业}研究生。一批和二批的流程都是笔试+面试,一批的报考要求会相对高一些,大部分会要求本科以上,二批次电工类专科也可以报考,岗位也会有相应的不同。注意,\textbf{有些省份的一批不招计算机专业的}(如重庆市),那错过了提前批,就只能去报二批。同时要注意,报考\textbf{基本上都有年龄要求},一般是要求,专科生不超过23周岁、本科生不超过25周岁、硕士研究生不超过28周岁、博士研究生不超过33周岁。
\subsubsection{提前批}


提前批主要是省公司及其分支机构才会有,一般在每年的 9-10 月份会在国网招聘官网上发布(首页-省公司-单位一览-最近动态),上面除了招聘信息外,还会列出校园宣讲会的行程表,如 24 年辽宁省的行程安排如图\ref{宣讲会行程}所示。


\begin{figure}[htbp]
    \centering
    \includegraphics[trim=210 0 210 0, clip, width=\textwidth, height=\textheight, keepaspectratio=false]{img/roadshow.pdf}
    \caption{24年校招辽宁电网宣讲会行程}
    \label{宣讲会行程}
\end{figure}

招聘团队每年会按照这上面的行程,到不同的城市和大学去进行宣讲,一般是当场收简历,当场面试,甚至是当场发offer。对于计算机类专业和通信类专业(信息学院的同学也可去报)的同学,给的offer岗位一般是各省的信通公司,以及市局的通信部门。

\textbf{要注意,并不是只有宣讲院校的同学才能去参加面试},所有符合条件的同学都可以去。所在城市没有行程安排,也要去别的城市参加。

省公司的提前批一般没有笔试,只有面试。面试跟之前介绍的泛体制面试一样,技术问题很少,有以下几个地方要注意。

1.如果在四六级,计算机等级考试,在校成绩,综合荣誉,和爱好特长方面有亮点的,在面试自我介绍的过程中,要加上这些内容,简历上也要写。


2.一定要表达出留在当地工作的强烈意愿。比如本地人想回老家;大学在那读的;想定居等等。


3.介绍项目/学习经历最好能往业务上贴,建议面试前,去网上了解一下目标省份信通公司最近在做的重点业务。然后这样去介绍,比如:我特别想去老家电网,平时也很关注家乡电网一直在做的工作,最近看到你们完成了xxx,那刚好我的研究方向是xxx,之前实习的时候做过xxx的相关项目,希望自己能有机会利用这方面积累的能力,为xxx的后续工作提供一份力量。

如果你在面试中失利,没有拿到心仪的offer,一个重要的建议是,继续去追宣讲行程。换位思考一下,如果你是面试官,发现有一个同学经常会在面试中出现,跟着宣讲在全国各地到处跑,是不是也会被他的诚意打动?这不一定百分百有效,但根据往届同学反馈的经验,有机会。

另外一个建议是,如果你的宣讲场次比较靠后,建议在时间方便的情况下直接去其他城市参加第一场宣讲,或者在前面几场中按照地点情况,挑一个竞争可能会小的场次。


\subsubsection{一批}

国网一批开启时间较晚,在秋招的尾声阶段,可以关注国网招聘的公众号,里面有详细的报名流程和简历填写指南。
在单位选择上,如果选择研究院要打听好该院的硕博比例,博士太多的话硕士可能就没有晋升机会了,还要关注一下出差的频繁度。如果是选择各地方的话,要看一下是否会被分到偏远地区(概率不小)。

\textbf{关于笔试:}

笔试一般是在周末,线下机考。包含少量行测、少量企业文化、大量计算机。

企业文化这个有固定题目,一般国网招聘公众号上都有。还记得当时有一道题问是国网铁军,还是国网雄狮,我觉得狮子比较威武就选错了,现在想起来还是很懊悔,这种企业文化题千万不能“你觉得”,是什么就是什么,背就完事了。

笔试其实很重要,占总成绩的百分之70,我当时面试表现很好,但是笔试裸考只有53分,最终没有拿到offer,所以得笔试者得天下。占大头的计算机,和考研408的风格不一样,有很多软件工程、数据库这类的东西,后来我发现相关的题库里都有,所以想去国网的一定要好好刷题,血泪教训。

\textbf{关于面试:}

笔试后的一周就是面试,各单位面试内容不一样,这里只说我报的某信通分公司。

一组10个人由老师带领候场,每次只进一个人,对面大概有5个面试官。首先面试官会翻一个八股的小册子,问一道简单八股,我被问的是错误码503代表什么出问题了。然后就会问项目经历,结合项目问更深层次的问题,我的感受是对面的面试官技术是可以的,问的问题甚至比有的大厂面试官还深还难,但探讨过程还是很友好的,这段在整个面试部分占了大部分时间。

然后应该是领导的人会问一些项目中用到的技术,这里说宏观框架就可以,说的过细会被打断。最后给你一段英文,一分钟准备,先朗读英文,再把他翻译成中文。面试就结束了,没有反问时间。







% \subsection{张昊-国家电网}
% \subsubsection{简历投递}

% 国网一批开启时间较晚,在秋招的尾声阶段,可以关注国网招聘的公众号,里面有详细的报名流程和简历填写指南。
% 在单位选择上,如果选择研究院要打听好该院的硕博比例,博士太多的话硕士基本就没有晋升机会了,还要关注一下出差的频繁度。如果是选择各地方的话,要看一下是否会被分到偏远地区(大概率)。

% \subsubsection{笔试}

% 笔试一般是在周末,线下机考。包含少量行测、少量企业文化、大量计算机。

% 企业文化这个有固定题目,一般国网招聘公众号上都有。还记得当时有一道题问是国网铁军,还是国网雄狮,我觉得狮子比较威武就选错了,现在想起来还是很懊悔,这种企业文化题千万不能“你觉得”,是什么就是什么,背就完事了。

% 笔试其实很重要,占总成绩的百分之70,我当时面试表现很好,但是笔试裸考只有53分,最终没有拿到offer,所以得笔试者得天下。占大头的计算机,和考研408的风格不一样,有很多软件工程、数据库这类的东西,后来我发现相关的题库里都有,所以想去国网的一定要好好刷题,血泪教训。

% \subsubsection{面试}

% 笔试后的一周就是面试,各单位面试内容不一样,这里只说我报的信通分公司(北京)。

% 一组10个人由老师带领候场,每次只进一个人,对面大概有5个面试官。首先面试官会翻一个八股的小册子,问一道简单八股,我被问的是错误码503代表什么出问题了。然后就会问项目经历,结合项目问更深层次的问题,我的感受是对面的面试官技术是可以的,问的问题甚至比有的大厂面试官还深还难,但探讨过程还是很友好的,这段在整个面试部分占了大部分时间。

% 然后应该是领导的人会问一些项目中用到的技术,这里说宏观框架就可以,说的过细会被打断。最后给你一段英文,一分钟准备,先朗读英文,再把他翻译成中文。面试就结束了,没有反问时间。
