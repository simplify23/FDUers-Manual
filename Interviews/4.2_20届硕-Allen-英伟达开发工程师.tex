\newpage
\section{20届硕-Allen-英伟达开发工程师}
联系方式:V:hustlmn

\begin{itemize}

\setlength{\parindent}{2em} 
    \item \textbf{可以介绍下您毕业后的工作经历吗?}

    A: 我是20年的毕业生,本科是能源与动力专业,研究生是跨专业考的计算机专硕,但我在校期间做了一些NLP算法项目的比赛,因此我的第一份工作是在某大厂做NLP算法岗位。

但是第一份工作工作了半年就裸辞了,并花了三个月自学了开发技术,由于我是0基础转行,因此学习路径主要是看网课学习,然后刷题。找开发工作时我面试的第一家公司就拿到了offer,是国内的一家中厂互联网公司,岗位是前端工程师。

在第二家公司干了一年后,又跳槽到了一家知名外企公司(英伟达)工作,是做的全栈开发,一直干到现在,大约有两年的时间。

    \item \textbf{第一份大厂算法岗为什么工作了半年就裸辞了呢?}

    (1)工作强度很大。公司每天晚上都要加班到十点或者十一点钟,一周要上六天班,就连法定节假日放假也有限制。可能对于有些人而言,晚上加班效率更高,比较能适应。但对于我而言,由于每天的精神状态都很紧绷,因此到晚上八点钟大脑就一片空白,甚至在晚上12点回家后,我仍然没有从工作的状态中抽离出来,就连做梦都会梦到工作中的事情,最后也影响到了身体。

(2)算法需要用不确定的结果去满足确定的业务目标。我个人认为,算法本身的精度就是一个概率问题,算法本身是不确定的,而工作中的算法技术是要落实到项目中的,因此对精度要求很高。因此我在做算法的过程中压力很大,也很痛苦。
既然工作岗位和工作作息我都不喜欢,因此就裸辞了,想要找到一份自己能接受的工作。


    \item \textbf{第一份工作工作强度这么大,薪资待遇是不是比较丰厚?}

    确实很高,我也很满意
    
    \item \textbf{为何学长要从算法转开发呢?}
    
    主要看每个人的兴趣。开发对我而言更如鱼得水,并且开发的结果是确定的,你知道只要按照这个路线做下去,你百分百能把东西做出来,但是算法更像开盲盒,哪怕花了很大精力,结果也未必有效,而且是那种精神上的压力。
    
    \item \textbf{可以分享一下第二份工作经历、工作强度和薪资待遇吗?}

    第二份工作是从算法跳槽到了中厂做开发,薪资少了一些。

我们的工作时间是大小周,一开始是晚上七、八点下班,但后来变成了九、十点钟下班。其实互联网公司组与组的差别很大,并且百分之八九十都是要加班,而且加班风气也很严重,这也是我要辞职并且找外企的原因。

在这一年工作中我的进步很大,毕竟从零开始学,也学了不少东西。在工作中最大的困难不是技术,而是沟通问题。如果对面的人技术能力和沟通能力都比较强,工作起来会很愉快。但如果对方加班很累,精神状态不好,当你提了一个需求让对方改,对方的情绪会很不好,这样的环境对我也有一些影响。

    \item \textbf{第三份外企工作可以做到work life balance吗?可以分享一下吗?}

    第三份工作薪资比第二份高了不少,而且我们部门相对比较轻松,例如可以居家办公,甚至不用打卡,还挺不错的,差不多是965的工作强度。但是有些部门也不轻松。

我在外企目前工作内容是全栈开发,前端后端都有涉及。外企不会要求你会特定的编程语言,他们更在乎的是你的学习能力。如果有同学想面试外企,需要着重突出你的学习能力。我的manager在录用我时和我说,看到我的简历觉得我的学习能力特别强,这也是看重我并录用我的原因。

    \item \textbf{在外企工作晋升的通道路径、薪资上限都是什么呢?}

    外企薪资构成主要分为工资和股票,工资每年会涨一部分,具体涨了多少需要根据市场行情来定,每年涨幅不一样。

对于股票部分,会每年给你发一部分股票,股票也是分几年给的,不是一次性给的。

工资是可以预测到的,例如你到多大年纪,能赚多少钱。但如果你想通过工资得到很高的薪资水平,不仅要非常努力,还需要有很高的天赋和运气,再结合领导的赏识,才能快速升职,我认为这个很难。所以需要一些不确定性因素,比如股票。外企股票不确定性很强,如果运气好,可以得到很高的报酬。

晋升通道方面有两条:一条是engineer线路,一条是manager线路,engineer主要做技术,而Manager的工作主要是规划项目和分配工作。想要进入manager线路,需要在engineer路线上达到很高等级,才能进入manager较低的等级。但是两者薪资差不多,主要看自己的性格和兴趣适合走哪一条路。

裁员的话,有的外企裁员,有的不裁员,我们公司目前没有说有裁员的动作。

    \item \textbf{有感受到大环境裁员风险带来的压力吗?要如何应对?}

    我认为不管在哪里上班,一定不要抱着稳定的心态来面对,唯一稳定的就是不稳定性,即便是公务员也没有我们想象的那么稳定。

首先要改变的是心态问题,我们要有信心即便被裁员了,也能找到很好的工作,如果国内找不到,也可以去国外找。

其次多在工作中积累项目,保持持续学习的能力。即使这个项目不是你做的,你也可以做到将这个项目从整体架构到实现细节完整的讲出来。并且定期三个月或者半年就去更新你的简历,这样可以督促你进步。

    \item \textbf{那算法转开发难度如何?可以分享一下经验吗?}

    我觉得难度在于你想不想。只要你有决心,你就能转。如果你认为不能立马进入到一个大公司,你可以先去小公司历练一两年,再去大公司。只要你坚定这个想法,并且去行动,就没有问题。

公司其实不在意你上一份工作的岗位,在意的是你身上的某种特质,所以完全不要给自己设限。


    \item \textbf{这个行业未来主要发展趋势?在政策环境和AI技术的影响下行业如何变化?}

    这个问题有点大,我感觉环境对各个行业的影响会越来越大,例如微软将做AI和云计算的员工搬迁到了美国和澳洲。但是我个人认为,大家可以保持心态上的稳定,允许一切事情的发生。


    \item \textbf{现在大模型这么火爆,你有没有后悔从算法转开发呢?}

    我一直不后悔,因为所有经历都是值得的,并且工作也讲求缘分。我其实一直不太相信AI,这过几年可能又是一片泡沫的状态。
    
    \item \textbf{在找工作时可以通过哪些就业渠道获取相关就业信息呢?}

    (1)公众号或者小程序。Offer show,或在公司的官方公众号里投递。

    (2)公司官网投递。

    (3)招聘app,如脉脉、牛客等。
    
    (4)小红书。我发现小红书也很好用。

    \item \textbf{如果有计算机专业的在校生想进入外企工作,要做哪些准备呢?}

    (1)多去实习。大家要是以找工作为目标的话,就多去实习,在学校阶段去试错。我之前之所以跳槽比较频繁,是因为我没有过实习经历,也不知道自己在工作中能做成什么样子。如果能多去实习就能知道自己更想要什么工作。

(2)狠刷力扣题。有些公司只要笔试过了,基本就没问题了。

(3)表达能力要清晰有逻辑。面试时,要清晰的表达你的意思,并且项目这个东西是可以准备的。我们同学之间可以相互模拟面试,当对面是小白时,给他讲一下项目,看他是否能听到项目的重点,并且听的明白。


    \item \textbf{如果我想从国内互联网企业跳槽到外企,有哪些心得吗?}

    (1)搜集外企名单。我的简历投递没有找内推,当时搜集了一些上海的外企名单,并且在公司官网上投递。

(2)刷英文力扣算法题。外企的笔试题目是英文的,并且还需要一定的引文口语表达能力。但是大多数情况是让你用英文介绍一下你的项目,介绍完后就会转成中文问项目。所以大家可以有针对性的去准备。例如用英文描述一下你的简历、思考对面可能会问到的项目问题,将其翻译成英文。

(3)对自己的项目了如指掌。我工作的这家外企很重视项目,从项目整体架构,到项目很细的点,都会问到,你要保证对项目介绍的很清楚,问到的细节也都能回答上来。

(4)心态稳定。面试很讲究运气,不要因为面试没过就难受,这说明这家公司与你无缘。并且再面试过程中我们对面试官的感受和印象很重要,毕竟这些人以后要成为你的同事,如果觉得很难受的话要慎重考虑。

    \item \textbf{有些人想先进入外企工作然后跳槽到其他国家工作,路径可行吗?}

    国内的外企很难,甚至可以说基本没有。我身边也有一些个例,例如有从亚马逊跳槽到美国的,也有直接面试去新加坡的tiktok等。但是薪资方面,国外要比国内翻了三倍左右。

对于绿卡,一般公司会先给你一个五年的工签,五年之后可以帮忙抽绿卡。不过也有可能运气不好,几年也没有抽中,但是也有办法解决。

不过有的公司会有转到国外的相关政策,例如亚马逊会有内部的招聘网站,如果你想去,就需要你达到特定的等级才能申请。对校招毕业生而言,需要工作两三年才能达到这个等级,假如申请上后,还要重新面试。

你也可以在平时工作中问问国外的工作伙伴有没有岗位和机会,说不定就掌握了一手信息。如果你有想法,去尝试就行了,失败了再说。


    \item \textbf{如果能重来一次校招,你会怎么做哪些改变?}

    我应该不会做哪些改变,基于我当时的认知水平和知识,应该不会有太大改变。
    
    \item \textbf{对那些迷茫于找工作的在校学弟学妹们,您有什么寄语吗?}

    我想说,人生是场马拉松,工作只是人生的一部分,它与学习是两码事。不要因为眼前工作的不好就认为人生灰暗了。未来还有几十年要工作,就算现在不好了,再过几年,未来总会变好的。

选择远远大于努力,所以大家一定要好好选择。

\end{itemize}
