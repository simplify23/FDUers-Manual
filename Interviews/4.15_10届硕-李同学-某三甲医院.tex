\newpage
\section{10届硕-李同学-某三甲医院}
\begin{itemize}

\setlength{\parindent}{2em} 
    \item \textbf{可以介绍下您毕业后的工作经历吗?}

最初工作在某三甲医院信息处,后因工作需要,岗位调整至人力资源处。

    \item \textbf{目前薪资待遇和工作强度是怎么样的呢?}

目前薪资待遇为三甲医院行政人员工资,执行事业单位基本工资与绩效工资标准。


    \item \textbf{您当初选择这份工作,从事这个行业的理由是什么呢?}

选择医院行业,是因为择业时,个人认为计算机专业在医院信息化过程中将发挥十分重要的作用,特别是医院信息系统项目应用方面。

  
    \item \textbf{这个工作/行业有哪些最令人满意的地方?}

最令人满意的地方是行业的稳定性。


    \item \textbf{这个工作/行业有哪些最想吐槽的地方?}

医院在计算机技术应用方面需求很大,但医院信息部门工作人员相对较少,无法完成自行开发,从而会选择软件服务商进行项目外包,计算机技术专业入职后,成为信息系统的问题解决协调者,专业发展受限。

    \item \textbf{这份工作/行业带给您最深感受/影响是什么?}

选择专业,应当以专业选择专业,专注于做专业的事情。无论是什么时候,人力资源部门首先会问你是学什么专业的,然后才能匹配什么样的岗位,提供相应的专业类别工资水平。

    \item \textbf{在你的行业中,职业晋升的通常路径是什么?有哪些职业发展的方向、机会或障碍?}

就事业单位来讲,职业晋升的通常路径遵循专业技术系列,如助理工程师、工程师、高级工程师、正高级工程师。晋升过程中,会要求一定的年限和承担项目,并在项目完成过程中在国内外核心期刊上发表专业论文。软件工程专业可以通过软件工程与技术考试,取得助理工程师、工程师、高级工程师、正高级工程师等职称。

企业类,除团队的MANAGER、CTO等职务系列外,建议也参照事业单位,参评专业技术系列职称。

    \item \textbf{行业近年来的主要发展趋势是什么?您预测未来几年内,(在政策/AI技术等因素的影响下)行业将会如何变化呢?}

信息已经深入在每个行业的社会经济生活中,无论选择哪个行业或者公司,只要在这个公司能够坚持工作5-10年,都可以形成产品优势和技术优势,在竞争与学习中保持核心竞争力。

    \item \textbf{对于计算机专业的在校生,如果将来想要从事这个行业,找到这样的工作,需要做哪些准备呢?}

参加国家、上海市科委、经信委,各级科委、信息委举办的信息发展大会或论坛,根据所学专业及导师研究方向,了解各类行业需求,找到自身的努力兴趣点,在查阅行业发展与趋势文献报告后,与导师一起探讨深入研究与应用。

    \item \textbf{您在找工作过程中通过哪些渠道获取相关就业信息呢?}

各高校招聘会、各就业网站、目标企业网站等。

    \item \textbf{可以复盘下您在校招时的经历和心得吗,如果能重来一次校招您会做哪些改变呢?}

按照所学专业,选择并坚持专业方向。

    \item \textbf{对那些迷茫于找工作的在校学弟妹们,您有什么寄语吗?}

迷茫是因为对未来事物的无知,多听、多看、多参与,常思、常研究、常交流。将他人走过的路,变成自己前进的方向,别人披荆斩棘,咱也继承发展,坚持学习并输出,终会走向成功。

\end{itemize}