\newpage
\section{20届硕-王先生-某头部券商程序化研究员}
\begin{itemize}

\setlength{\parindent}{2em} 
    \item \textbf{可以介绍下您毕业后的工作经历吗?}

刚工作一年,一开始是协助高级研究员进行市场趋势分析、因子分析或初步的量化策略开发,后续进行独立开发量化交易策略,涵盖股票、期货、期权等多种金融工具。

    \item \textbf{目前薪资待遇和工作强度是怎么样的呢?}

目前薪资水平高于同行,没有固定加班,加班算调休但是一般调休不了。强度的话随时待命吧


    \item \textbf{您当初选择这份工作,从事这个行业的理由是什么呢?}

本科就是学习的金融,但是不希望直接从事金融行业,因为这个行业是随着整个金融市场的周期进行波动的,随着疫情,战乱等出现,世界局势不明朗,这个行业的前景也不明朗,所以考虑做技术和金融结合的方面。

  
    \item \textbf{这个工作/行业有哪些最令人满意的地方?}

薪资比较满意,周围厉害的人比较多。


    \item \textbf{这个工作/行业有哪些最想吐槽的地方?}

越来越对钱的多少敏感,周围奇葩的人际关系和事情还是挺多的,感受到世界的多元化和阶级的固化。

    \item \textbf{这份工作/行业带给您最深感受/影响是什么?}

不要做投机的事情,不要炒股,不要炒股,不要炒股。

    \item \textbf{在你的行业中,职业晋升的通常路径是什么?有哪些职业发展的方向、机会或障碍?}

路径的话就是初级,中级,高级研究员,后续就是部门负责人和首席。障碍的话身体可能需要注意,之前同事心脏骤停差点没救回来,对我的影响挺大的,保重身体就是最大的职业财富。

    \item \textbf{行业近年来的主要发展趋势是什么?您预测未来几年内,(在政策/AI技术等因素的影响下)行业将会如何变化呢?}

趋势是AI和人工的结合吧。国内正式步入大通胀时代,CPI数据持续不明朗,伴随着经济下行,金融行业其实在风口浪尖上。从未来回顾历史,可能在去年Signature Bank倒闭之后,就代表着2023的金融危机已经开始。在美联储不断加息,国内内需不振的今日,这个行业的首要目的应该是活下去。

    \item \textbf{对于计算机专业的在校生,如果将来想要从事这个行业,找到这样的工作,需要做哪些准备呢?}

看看有没有金融背景,没有的话,金融知识,数据分析知识要多看看,其他的,先从Quantopian、Interactive Brokers 等基础的东西开始练手吧,最好能在实习的时候进来,毕竟在学校学习的和实际用到的差别挺大的

    \item \textbf{您在找工作过程中通过哪些渠道获取相关就业信息呢?}

我是直接熟人推荐的

    \item \textbf{可以复盘下您在校招时的经历和心得吗,如果能重来一次校招您会做哪些改变呢?}

因为论文的硬性要求,所以找工作的时间比较紧,基本就是没熟人推荐的话多刷刷题,重视项目而非代码细节,注重实现流程而非实现结果。有熟人推荐的话就是早一点熟悉,融入团队,好的沟通能力比编程能力要重要很多。

    \item \textbf{对那些迷茫于找工作的在校学弟妹们,您有什么寄语吗?}

一切事情在解决之后进行回顾都会发现它微不足道,保持好心态,同时沉下心来,做事不要心向大海,因为涛声有望而不可及。

\end{itemize}