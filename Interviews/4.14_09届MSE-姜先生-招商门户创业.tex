\newpage
\section{09届MSE-姜先生-招商门户创业}
\begin{itemize}

\setlength{\parindent}{2em} 
    \item \textbf{可以介绍下您毕业后的工作经历吗?}

复旦本科化学系毕业后,02-06年留校在学校宣传部网络宣传办公室,负责学校主页、新闻网等,偏技术和管理角色。后来进入互联网公司Lightinthebox和新浪微博商业产品部,转互联网产品方向。13年从微博离职后创业,创办望才大数据招聘,16年初被智联招聘并购。后又主导或参与区块链、在线母婴医疗等创业项目。21年开始做政府产业招商的大数据产品和平台,目前主要是在做“人人招商”和“招商门户”这2个微信小程序。

    \item \textbf{目前薪资待遇和工作强度是怎么样的呢?}

之前在大厂做互联网产品的薪资是比较高的,最高的时候是80K/月。当前算创业阶段,给自己开很低的基本工资,15K左右,今年预计公司会有1000-2000万左右的收入,大约会有40\%-50\%的净利润,但大概率会继续投入而不是分红。互联网领域996很普遍,但是现在卷过之后,也倾向于谋定而后动,尽量让公司和团队往正确方向发展,减少无谓的加班和内卷。


    \item \textbf{您当初选择这份工作,从事这个行业的理由是什么呢?}

选择产品方向是因为热爱,被乔布斯影响很深。选择离开复旦和创业都是因为自己想做正确的事,更加自由,不受太多的束缚,不希望把青春浪费在一些官僚或者错误的上级指令上。

  
    \item \textbf{这个工作/行业有哪些最令人满意的地方?}

自由,可以按照自己的想法来经营公司,发展产品和服务。政府产业招商也是个万亿级市场,过往比较传统,现在有机会用人工智能和大数据做一些数字化方面的提升和改造。商业模式也有很大的创新机会,令人兴奋。


    \item \textbf{这个工作/行业有哪些最想吐槽的地方?}

过往太依赖于线下人脉,无论是政府关系还是企业项目项目的获取。得花非常多的时间和精力才能优化一点点,改变从业者的认知,步履艰难。

    \item \textbf{这份工作/行业带给您最深感受/影响是什么?}

个人对影响一件事成功与否的认知上,从过于依赖互联网产品+技术,会逐步切换到运营、市场,乃至于销售上。创业确实要求多面手,或者得用合伙人团队来补足个人的短板。

    \item \textbf{在你的行业中,职业晋升的通常路径是什么?有哪些职业发展的方向、机会或障碍?}

公司内部晋升,主要看情商,而不一定是能力,除非你的工作能力远超常人,例如能给公司带来销售大单,技术上不可或缺等。跨公司的晋升,可以通过跳槽。职业发展主要要选好行业,入错行再调整,至少浪费1-2年时间。然后是跟好老板,如果能认识一个好老板,则职业发展会顺利很多。好老板的标准是:愿意带你(信任),他自己在公司和行业内的成长也很快(潜力),能快速学到新东西(成长速度)。但是好老板为什么要提携你,而不是另外一个同事,这个就是我说的情商。

    \item \textbf{行业近年来的主要发展趋势是什么?您预测未来几年内,(在政策/AI技术等因素的影响下)行业将会如何变化呢?}

AI对各行业的影响确实会越来越大,对于计算机系毕业生而言,最大的机会在AI算法上,例如大模型等。应用层目前可能才算刚刚开始,也有很多机会。
另外中美脱钩、贸易战等也会改变整个世界竞争合作的方式,包括目前美元基金退出,人民币基金主要由政府主导等。所以硬科技是目前的主要投资方向。建议关注国家对战略新兴产业的发展导向,里面也会有相应的机会。

    \item \textbf{对于计算机专业的在校生,如果将来想要从事这个行业,找到这样的工作,需要做哪些准备呢?}

不建议毕业即创业,成功概率太低。咱们复旦计算机的毕业生,找工作可以找大厂作为起点,站在巨人的肩膀上先积累行业经验和人脉,未来的路会非常好走。

    \item \textbf{您在找工作过程中通过哪些渠道获取相关就业信息呢?}

通过招聘平台或者校招等渠道当然可以,但是能提前到大厂实习可能就更好,先人一步。本质上这个问题的核心是有没有办法,提前或者更多的和一些大厂的人结识,这样内推的机会也多,入职概率也高。师兄师姐会是很好的资源,建议多加入校友群,多参加校友活动,多与人聊天沟通。

    \item \textbf{可以复盘下您在校招时的经历和心得吗,如果能重来一次校招您会做哪些改变呢?}

a、想办法构建自己的差异化价值,能体现在简历上,让人眼前一亮

b、多花时间在人脉网络构建上

c、选择发展潜力大、天花板更高的行业

    \item \textbf{对那些迷茫于找工作的在校学弟妹们,您有什么寄语吗?}

相信自己,乐观开放积极。无非现在的状态是潜龙在渊,但是见龙在田、飞龙在天不会远的。

\end{itemize}