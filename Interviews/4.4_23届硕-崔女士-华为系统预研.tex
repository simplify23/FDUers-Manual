\newpage
\section{23届硕-崔女士-华为系统预研}
\begin{itemize}

\setlength{\parindent}{2em} 
    \item \textbf{可以介绍下您毕业后的工作经历吗?}

    毕业之后就入职华为ICT产品线做系统预研。
    
系统预研工作内容可以分为三个方向:

第一个方向是围绕着产品系统的一些硬件技术或者软件技术,网络协议等未来五到十年的一个演进路径,涉及对整个系统架构的一些调整,我目前的工作内容主要是结合整体的方案设计,配合团队做一些细粒度方向上的设计。

第二个方向是对既有的产品系统进行探索优化,希望通过方案设计把性能达到最优化,确定优化的模块,围绕设计方案进行性能验证,如果最后能证明这个性能方案对系统可以实现一个性能提升,就可以落地交付。

第三个是围绕最近热门的研究方向做一些洞察研究,比如说AI大模型,找到可以和产品结合的点。

    \item \textbf{目前薪资待遇和工作强度是怎么样的呢?}

压力是可以接受的。在华为如果是比较贴近业务交付的方向,比如说软件开发这种工作强度会更大。预研的性质和开发交付不太一样,交付周期相对长一些,从设计到落地到性能验证、开发时间周期长。对知识的深度和广度有一定的要求。压力更多的是看自己对自己的科研能力的一个要求。

工作强度取决于当前项目节奏。周六周日的加班情况是基本上不加,一年半总共就加班了四次(除去月末周六)。
薪资方面是保密的,不过和互联网大厂相比来说大概和美团的SSP是差不多的。


    \item \textbf{这个工作/行业有哪些最令人满意的地方?}

    大体我是很满意的,一方面我认可系统预研这个方向,未来的方向是AI大模型,对训练来讲是数据为基础的。国家也在推进构建数据中心,数据中心会成为国家的一个基础建设,这方面华为也在努力推进。
    
另一方面我不想做纯技术的方向,我希望每个阶段有一些新的内容,新的方向可以学习。我现在的部门工作模式相当于每三到五个月是一个周期,在这样的一个周期里可以去研究新的内容。比如通过研究调度层,你会把整个底层有一个了解,最终对整个系统的了解就会构建成个人的技术栈。我们部门是做整个产品生态的演进方向,所以对整个系统架构都需要有了解,同时也会结合最新的产品市场需求去定位未来发展的方向。工作自由度很高,强度也可以适应。
  
    \item \textbf{跳槽是很多同学会选择获取快速涨薪的一种方式,就你观察系统预研方向跳槽的前景如何?对应届生来说有裁员压力吗?}

    华为是有内转的,内部流动速度很高,内部是可以实现通过内转来实现职级的提升。

华为内部还是相对稳定的,因为对新员工的培养成本是很大的,导师制度和NEO培训制度会保障不会让一个新员工容易被裁掉。除非是这个人和预期差距过大。但是会有同学在培训期间因为和企业价值观不符合而主动离开的。

    
    \item \textbf{在你的行业中,职业晋升的通常路径是什么?有哪些职业发展的方向、机会或障碍?}

华为有一套自己的晋升体系,晋升标准明确。

从13级到14级,从14级到15级可能会相对于15级到16级会快一点。但是主要是看是否有工作承重,能力是否达标。


    \item \textbf{行业近年来的主要发展趋势是什么?您预测未来几年内,(在政策/AI技术等因素的影响下)行业将会如何变化呢?}

    系统预研目前来看是处在上升阶段,国内在这个方向的发展是要比国外好的。国家层面来说,数据中心的部署已经在路上了,大模型的数据需求量很大,如何对这些数据进行管理,保证数据的安全性都是一个比较大的挑战。随着数据中心的不断扩展,挑战肯定也是越来越大的。

华为在这一方向的布局,是顺应时代的发展,顺应时代的变化。华为对研究方向的选择是很慎重的,所以这个方向肯定是有一定的价值的。

    \item \textbf{对于计算机专业的在校生,如果将来想要从事这个行业,找到这样的工作,需要做哪些准备呢?}

好好准备简历,把项目准备清楚,心态要放轻松。

其实找工作也是很看运气的,我刚好在面试的时候遇到了一个非常聊得来的主管,另一方面上研这边涉及一个人员的流动,会变成一个重点基地,机遇比较好。而且我所研究的方向比较契合。

    \item \textbf{您从面试到拿offer的过程中和hr是怎么沟通的呢?怎么判断自己是不是在泡池子呢?}

从面试到拿offer期间,每个月保温一次。而且是可以通过接口人的态度和主管面来判断自己究竟可不可以拿到offer。我当时主管面和主管就聊得很开心,对拿到offer的把握较大。

    \item \textbf{对那些迷茫于找工作的在校学弟妹们,您有什么寄语吗?}

相信自己,不要焦虑。不管是研一还是研二,把当前手里做的事情做清楚,做明白就挺好的了。

以前其实我不想再去做AI,我觉得它技术迭代太快了,我不想经过很久之后我的技术会因为它迭代的问题而达不到一个积淀。但是我现在所在的这个部门,虽然更偏向开发研发这个方向,但是其实我们也会结合当前大模型这个比较火热的背景去做一些融合,不一定说每一个阶段做的东西就真的无用了,每一步都作数,把当下的事情做好。

\end{itemize}
