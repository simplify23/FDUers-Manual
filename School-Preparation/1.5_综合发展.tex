
\section{本科生综合发展}

信息是取得综合发展的前提,多关注校园内公众号,多认识不同专业同学,会有机会了解到更多综合发展的机会,比如一些含金量较高的活动比赛等等。多认识好的老师,参与他们组织的活动,他们会为你提供更多高质量机会。

\subsection{入党} 
如果想要入党,大一提交入党申请书,经过积极分子考试和发展对象考试,大概需要四年时间。每个班级入党名额都很有限供不应求,相对来说会优先考虑团总支书、班长、团支书等岗位的同学,其次就是参与志愿服务较多的同学,另外也跟你在积极分子评选大会上的发言质量、你平时跟同学的关系有关。想要入党及早准备即可,同时考试前认真复习(如果挂科就需要下学期补考),其他没有太多需要注意的问题。
\subsection{学生工作}
学生工作的意义见仁见智。如果是要考虑选调工作等,学生干部会有加分。整体来说学校里面的学生工作可大致分为分为校学生会、校团委、院系分团委学生会、书院以及社团、团艺等,校级层面的学生工作可以认识不同专业同学,但是不同部门的氛围工作环境等差异极大,院系层面可以结实同学院不同专业同学,尤其是可以形成跨年级社交,高年级学长学姐可能是你下一门课程的助教,你也可以跟他们索取一些学习资料和经验。学生工作的招新在复旦校学生会、团团在复旦、腾飞书院、计苑菁英等工作号每年春秋季都有招新通告。

学生工作的晋升路径大概是部员-副部长-部长-主席,做到主席对于之后校园内各项评奖评优会有加分,同时对于自己的组织沟通能力会是很好的锻炼,如果对于自己的能力比较自信,有敢于拼搏的精神,可以尽可能勇敢尝试。最辛苦的是部长层级,操所有的心落实所有的工作。

整体而言如果是想要走学术道路不建议在学生工作方面花费过多心思,可以尽早去实验室看论文发论文,会为自己赢得更多发展机会。

\subsection{志愿活动}
复旦提供了非常多参与志愿服务的机会,包括很多线上线下支教、关爱社区老人、图书馆整理等等。想要参与志愿活动可以多关注一些相关社团的公众号,比如“拓客社”、“远征社”、“彩云支南”以及“复旦团委青志”等等。参与志愿活动可以培养一个人利他的情怀,同时认识很多同样热心关爱他人的人。对于参与评奖评优也有一定帮助。

\subsection{社会实践}
每年暑假、寒假、春季和秋季学期都有立项进行社会实践的机会,可以关注“复旦团委实践”公众号,社会实践立项很简单,签字可以找辅导员。社会实践主要要有想法,可以多借助gpt,同时多留心生活中的真实需求。功利地看,社会实践只要经过简单包装很容易评为十佳或者优秀。

\subsection{人才培养计划}
人才工程计划,保研可以加分但是要做四年辅导员,在大三的时候有人才工程预备队“青年复旦”可以进行报名;研支团计划,即外出支教一年,然后回来读研,同样保研可以加分,大三的时候有研支团预备队“笃行计划”可以报名,这两个计划都可以认识一些复旦青年才俊。此外还有思源计划,复旦会倾注非常多资源进行培养,有免费出国机会,但思源计划选拔极为苛刻,有多轮面试每年从4000多名本科生中筛选约30人。此外新生有新生骨干培训班,研究生也有。另外还会有一些党团培训班等。