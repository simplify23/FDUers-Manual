
\section{信息获取}
\subsection{校内}

\begin{itemize}
    \item 复旦大学就业指导中心(公众号:复旦生涯)
    
    这个公众号属于复旦官方对学生就业的指导,当然这个公众号也会发布企业招聘、实习以及宣讲会信息,更重要的是在这个平台上可以获取就业时的各种手续上的文件与解答,可以直接公众号预约线下,也可以直接在公众号上获取就业、选调、国际组织等多个方面的信息。官方的复旦学生就业信息获取的网站为:https://career.fudan.edu.cn。
    
    \item 复旦大学基层就业服务协会(公众号:复旦基层就业)

    这个公众号对准备参加选调考试的同学来说非常重要。选调报名通常需要提交一份推荐表,而这张表格需要学校盖章。盖章时间通常是统一安排的,具体时间会在这个公众号上公布,因此建议大家密切关注。此外,这个公众号还会发一些选调宣讲和选调政策分析,每年也会为各省的选调考生建立复旦考生交流群。群内不仅会发布一些学校的通知,同学们也会在群里相互组织模拟面试,进行对练。


    \item 复旦ijob协会(公众号:复旦ijob协会)

    这个公众号是一个就业信息获取平台,公众号的每条资讯都是不同企业的招聘信息,公众号更新活跃,几乎每天都有新发布,可以关注该公众号获取及时的招聘信息。    

\end{itemize}

其中在本校就业信息网可重点关注地方选调和部委特招的信息。此外可以多联系学长学姐,加入校友或者院友群,群里面不定期会发布一些内部招聘,或者实习信息,可以作为拓展人脉和获取招聘/工作内容等信息的渠道。

依照经验来说,在找实习或春秋招时,首先会时刻留意心仪公司的官网消息,这是最“官方”的方式,其次可能就是在就业群中突然发现了一个感兴趣的实习或项目等,就去联系内推或面试了,所以加一些复旦学生组织的求职群是一个不错的途径,不时留意一下其中的消息。

\subsection{互联网平台}
\begin{itemize}
    \item 公众号渠道(最方便快捷):1.例如各大高校就业公众号(学校就业网站);2.校招信息大全,例如漫步五角场、OfferShow等。这一类微信群通常需要花费几十元购买一份招聘信息汇总表,购入后会拉一个微信群作为每日招聘信息发布和沟通的渠道。各家的求职汇总和沟通良莠不齐,建议多参与几家进行信息拼接缝合和比对;3.微信搜索公众号"目标企业名+招聘"。
        
    \item BBS:部分高校的BBS较为活跃,尤其是北大BBS,它专门有一个找工作的板块,上面会有很多高质量的选岗分析贴,面临offer选择时,可以借同学一个号上去问问,有些学长会提供一些珍贵的信息和犀利的观点。
    \item 牛客: 这个肯定会用到,有专门的求职专栏,提供笔面经和薪资爆料信息。
    \item 小红书:通过小红书上的相关帖子,求职者可以加入同一公司或同一求职方向的群聊,这为有相同目标的同学提供了一个沟通交流的便利渠道。不过,根据我的经验,小红书上有一些从事求职辅导的账号会发布不实内容,因此在获取信息时必须加以甄别。相比之下,个人账号的分享通常更真实可靠,建议多参考这些信息贴,并尝试直接联系分享者,以获取更准确的求职信息。
    \item 当地人才招聘:直接百度搜索,注册后可以直接在上面投简历,适合求职地点明确的同学。 
    \item 各类招聘软件:例如脉脉、boss直聘、拉勾网、智联招聘等。招聘软件对于校园招聘来说可能并不是最有效的方式,但是海投的话速度很快,不用重复填写个人信息,在上面放简历之后会收到各种猎头的电话,有一些机会还是不错的。
\end{itemize}
